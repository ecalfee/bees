\documentclass[12pt]{article}
\usepackage{nicefrac}

\begin{document}
	\title{Introduction}
	\date{}
	\maketitle

\begin{paragraph}
The spread of africanized honey bees out of Brazil formed moving clines in ancestry that spread North and South across the Americas. This wave of advance has slowed, if not stopped, and today we observe African-to-European ancestry clines at similar latitudes in North and South America. 
\end{paragraph}
\begin{paragraph}
What maintains theses clines? One hypothesis is that the continual influx of European honey bee gene flow from large apicultural centers has slowed the advance of Africanized honey bees (CITE). A second hypothesis is that these clines have formed at the climatic range limit for Africanized honey bees. African ancestry is associated with foraging for pollen over nectar (CITE), which increases reproductive rates at the expense of honey storage needed for winter thermal regulation and survival (CITE). Indeed, highly Africanized honey bees do not survive at **** latitude (CITE - personal communication, or citation possible from INTA?). Underdominance can also create clines, and although a priori we may not expect low hybrid fitness in crosses between such recently diverged subspecies (**** kyr CITE), studies have found that hybrid Africanized bees have excess wing asymmetry, an indicator of developmental missregulation (CITE). 
\end{paragraph}
\begin{paragraph}
How do we distinguish between these mechanisms of cline formation? Neutral blending of ancestries across 'contact zones' creates wide shallow clines. If selection maintains the clines due to local adaptation across an environmental gradient, we expect to see steeper transitions in ancestry relative to migration rates. Often the steepness at the center of the cline, which captures selection at linked loci, is compared to the steepness at the tails of the cline, which predominately captures neutral dispersal rate (CITE). In this system, we can also estimate dispersal rates using the well-documented historical spread of Africanized bees across the Americas, which averages ****km per year. 'Tension zones', or clines maintainted by underdominace, are also steep but distinguishable from clines maintained by local adaptation because their location is not tied to environmental gradients and instead tend to gravitate towards lower quality environments ('sinks') and/or geographic barriers to gene flow (e.g. mountain ranges) (CITE). Therefore tension zones would be less likely to show congruence between North and South America. In the Argentinian hybrid zone, we can also compare the cline position from a 1991 survey (Sheppard 1991) to the position today to test if the zone is stable. Of course, all three mechanisms may be contributing to the maintenance of these clines.

\end{paragraph}

	
	
\end{document}